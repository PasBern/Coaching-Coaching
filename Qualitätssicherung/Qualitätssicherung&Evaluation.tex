% !TeX encoding = ISO-8859-1
\documentclass[
10pt,			%Schriftgr��e
paper=a4,		%Papierformat
ngerman,		%Deutsche Silbentrennung
BCOR=0pt,		%Bindekorrektureinstellung=AN (15mm) Bindekorrektureinstellung=AUS > 0pt
%twoside,		%Bindekorrektur f�r zweiseitigen Druck
DIV=calc,		%Satzspiegelberechnung
headinclude,	%Kopfzeile bei Satzspiegelberechnung mit einbeziehen
headsepline,	%Trennlinie unter der Kopfzeile
numbers=noenddot,
]{scrreprt}

\usepackage[T1]{fontenc}
\usepackage{lmodern}
\usepackage[sc]{mathpazo}

% Satzspiegel wird neu berechnet
\typearea[current]{calc}

\usepackage[ansinew]{inputenc}

% Paket f�r besseres Schriftbild
\usepackage{microtype}


% Allgemeine Pakete:

\usepackage{xcolor}
\usepackage[ngerman]{babel}
\usepackage{relsize}
\usepackage{paralist}
\usepackage{mdwlist}
\usepackage{typearea}
\usepackage{setspace}
\usepackage{textcomp}
\usepackage{marvosym}

\usepackage{fancybox}
\usepackage{array}
\usepackage{amsfonts,amsmath,amssymb}
\usepackage{pifont}
\usepackage{calc}
\usepackage{ifthen}
%\usepackage{tikz}
%\usepackage{pgf-pie}
%\usepackage{pgfplots}
%\pgfplotsset{compat=1.8}


%\usetikzlibrary{shapes,arrows,shadows}

\usepackage{scrpage2}

% Pakete f�r Abbildungen
\usepackage{graphicx}
\usepackage{floatrow}
\usepackage{floatpag}



% Fu�notengefrickel muss vor HYPERREF geladen werden
\usepackage[bottom,flushmargin,hang,multiple]{footmisc}

%Hyperxmp-Paket einbinden f�r PDF-Metadaten
\usepackage{hyperxmp} 

% Rahmen um Hyperlinks unterbinden
\usepackage[bookmarks=true,pdftoolbar=true,pdfmenubar=true,pdfstartview={FitH},citecolor=magenta,hidelinks,breaklinks=true]{hyperref}



% Title Page
\title{Qualit�tssicherung \& Evaluation im Coaching}
\author{Pascal Bernhard}


\begin{document}


\maketitle

\begin{abstract}
Hier sollen Qualit�tsmessung, \&-sicherung und Evaluationsmethoden im Coaching beschrieben werden
\end{abstract}

\chapter{Qualit�tssicherung im Coaching}

\section{Einleitung}

Auf dem sehr heterogenen Coaching-Markt, dessen vielf�ltige Angebote bereits im Kapitel XXX erl�utert wurden, ist die Definition von Qualit�tsstandards und die Qualit�tssicherung unerl�sslich, um Coaching von anderen professionellen Beratungsformen zu trennen. Des weiteren helfen allgemeine und f�r auch branchenfremde Personen nachvollziehbare Standards diesem Beratungszweig zu mehr Akzeptanz und Anerkennung.
Zur Bewertung von Coaching-Ma�nahmen ist eine klare Definition, was Qualit�t im Coaching bedeutet unerl�sslich. Insbesondere die Vielfalt der Coaching-Angebote mit unterschiedlicher Zielsetzung und Vorgehensweisen steht einem eindeutigen und einheitlichen Qualit�tsverst�ndnis im Weg\footnote{Bischoff 2011, S.34}.

Die Versuche von Coaching-Verb�nden mittels Zertifizierungen die Qualifikation und indirekt hierdurch auch die Qualit�t von Coaching sicherzustellen, k�nnen nur bedingt als Qualit�tssicherung gelten. Ihre diesbez�glichen Kriterien wie zum Beispiel Berufserfahrung der zertifizierten Coaches, Umfang der Ausbildung, Teilnahme an Supervision und die Einhaltung ethischer Richtlinien lassen ein Aussage �ber die Qualifikation eines Coaches nach transparenten Ma�st�ben treffen. Jedoch nicht zwangsl�ufig �ber die Qualit�t des durchgef�hrten Coachings. Die Qualifikation ist zwar Voraussetzung f�r qualit�tsvolles Coaching, hiermit aber nicht gleichzusetzen. So ist erstens ein genau Definition, was Qualit�t im Coaching bedeutet erforderlich und zweitens die Frage zu kl�ren, wie Qualit�t gemessen und �ber die unterschiedlichen Coaching-Formate hinweg sichergestellt werden kann.



\subsection{Begriffsdefinition \textsl{Qualit�t \& Qualit�tssicherung}}


Allgemein betrachtet ist Qualit�t die Gesamtheit von Eigenschaften und Merkmalen eines Produktes oder einer Dienstleistung, die sich auf deren Eignung zur Erf�llung festgelegter oder vorausgesetzter Erwartungen beziehen\footnote{Kiefer \& Scharnbacher 1996, S.37}. Die im Bereich der industriellen Produktion entwickelte Qualit�tsdefinition und Qualit�tssicherung sollte jedoch nicht unreflektiert auf die Dienstleistung Coaching angewandt werden. 


\end{document}          
